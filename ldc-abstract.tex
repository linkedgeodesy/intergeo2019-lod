\documentclass[a4paper]{article}
%\usepackage{simplemargins}

\usepackage[
	pdftitle={Linked Data Cloud: Bindeglied zwischen Geod{\"a}sie und Gesellschaft},
	pdfsubject={Linked Data Cloud: Bindeglied zwischen Geod{\"a}sie und Gesellschaft},
	pdfauthor={Florian Thiery},
	pdfkeywords={Linked Data, Geod{\"a}sie 2030, Arbeiten 4.0}
]{hyperref}

% DOI and ARXIV Commands for Bib Files
% Written by Daniel Herber
% -----------------------------------------------
% one option is to use the 'note' field with this command
% -----------------------------------------------
% for example, if your doi is 10.2514/1.J052182
% then for the citation for the reference in your bib file, use
% note = "\doi{10.2514/1.J052182}",
% -----------------------------------------------
% for example, if your arxiv number is 0706.1234
% then for the citation for the reference in your bib file, use
% note = "\arxiv{0706.1234}",

% requires hyperref package for \href command
\usepackage{hyperref}

% doi command (use in bib file)
\newcommand{\doi}[1]{{doi:~\href{http://doi.org/#1}{#1}}\rmFullStop}

% arXiv command (use in bib file)
\newcommand{\arxiv}[1]{{arXiv:\href{https://arxiv.org/abs/#1}{#1}}\rmFullStop}

% command to remove full stop if the next character
\newcommand*{\rmFullStop}{\rmifnextchar{.}{}{}}

% command to check the next character and replace if present
% \rmifnextchar{X}{[removed text]}{[no X text]}
% if X is the next character, then it is removed and [removed text] is inserted
% otherwise, the character is not removed and [no X text] is inserted
% based on http://tex.stackexchange.com/questions/72827
\makeatletter
\newcommand{\rmifnextchar}[3]{%
  \begingroup
  \ltx@LocToksA{\endgroup#2}%
  \ltx@LocToksB{\endgroup#3}%
  \ltx@ifnextchar{#1}{%
    \def\next{\the\ltx@LocToksA}%
    \afterassignment\next
    \let\scratch= %
  }{%
    \the\ltx@LocToksB
  }%
}
\makeatother
%\RequirePackage{doi}
%\usepackage[square]{natbib}
\usepackage{amsmath}
\usepackage{amsfonts}
\usepackage{amssymb}
\usepackage{graphicx}

\begin{document}
\pagenumbering{gobble}

\Large
 \begin{center}
Linked Data Cloud:\\Bindeglied zwischen Geod{\"a}sie und Gesellschaft\\ 

\hspace{10pt}

% Author names and affiliations
\large
Florian Thiery$^1$\\

\hspace{10pt}

\small  
$^1$ R{\"o}misch-Germanisches Zentralmuseum - Leibniz-Forschungsinstitut f{\"u}r Arch{\"a}ologie\\
rse@fthiery.de\\

\end{center}

\normalsize

Geod{\"a}ten leben bereits im Jahr 2019 in einer globalen, digitalen und vernetzten Welt: der Cloud. Wir erleben das »Arbeiten 4.0« in einer »Industrie 4.0« mit dem Einsatz von cyber-physischen Systemen und der Vernetzung von Maschinen, Ger{\"a}ten, Sensoren, Menschen und Daten. 80\% aller Daten besitzen einen Raumbezug, somit sind  »Geodaten der Treibstoff der digitalen Gesellschaft«. Dar{\"u}ber hinaus werden Bestandteile des OOO-Modells\cite{ooo} immer wichtiger: Open Source Software, Open (Geo-) Data und Open Access. Zur Bereitstellung von offenen und freien (Geo-) Daten in interoperablen Formaten eigenen sich insbesondere Techniken des Web 3.0, sogenannte Linked Open Data\cite{tim_linked_2006} (LOD), bzw. Linked Open Geodata\cite{florian_thiery_2018_1421690}. Durch die Bereitstellung von LO(Geo-)D entsteht eine riesige vernetzte Linked Data Cloud\cite{lodcloud} verschiedenster Disziplinen (Geod{\"a}sie / Geisteswissenschaften / Naturwissenschaften), in der amtliche Geodaten als Bindeglied dienen, um neues Wissen f{\"u}r die Gesellschaft zu erzeugen\cite{florian_thiery_2019_2620929}.

\bibliographystyle{IEEEtran}
\bibliography{bib}

\end{document}