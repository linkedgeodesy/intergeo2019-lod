\documentclass[a4paper]{article}
%\usepackage{simplemargins}

\usepackage[
	pdftitle={Linked Data Cloud: Bindeglied zwischen Geod{\"a}sie und Gesellschaft},
	pdfsubject={Linked Data Cloud: Bindeglied zwischen Geod{\"a}sie und Gesellschaft},
	pdfauthor={Florian Thiery},
	pdfkeywords={Linked Data, Geod{\"a}sie 2030, Arbeiten 4.0}
]{hyperref}

\input{doiCmd}
%\RequirePackage{doi}
%\usepackage[square]{natbib}
\usepackage{amsmath}
\usepackage{amsfonts}
\usepackage{amssymb}
\usepackage{graphicx}

\begin{document}
\pagenumbering{gobble}

\Large
 \begin{center}
Linked Data Cloud:\\Bindeglied zwischen Geod{\"a}sie und Gesellschaft\\ 

\hspace{10pt}

% Author names and affiliations
\large
Florian Thiery$^1$\\

\hspace{10pt}

\small  
$^1$ R{\"o}misch-Germanisches Zentralmuseum - Leibniz-Forschungsinstitut f{\"u}r Arch{\"a}ologie\\
rse@fthiery.de\\

\end{center}

\normalsize

Geod{\"a}ten leben bereits im Jahr 2019 in einer globalen, digitalen und vernetzten Welt: der Cloud. Wir erleben das »Arbeiten 4.0« in einer »Industrie 4.0« mit dem Einsatz von cyber-physischen Systemen und der Vernetzung von Maschinen, Ger{\"a}ten, Sensoren, Menschen und Daten. 80\% aller Daten besitzen einen Raumbezug, somit sind  »Geodaten der Treibstoff der digitalen Gesellschaft«. Dar{\"u}ber hinaus werden Bestandteile des OOO-Modells\cite{ooo} immer wichtiger: Open Source Software, Open (Geo-) Data und Open Access. Zur Bereitstellung von offenen und freien (Geo-) Daten in interoperablen Formaten eigenen sich insbesondere Techniken des Web 3.0, sogenannte Linked Open Data\cite{tim_linked_2006} (LOD), bzw. Linked Open Geodata\cite{florian_thiery_2018_1421690}. Durch die Bereitstellung von LO(Geo-)D entsteht eine riesige vernetzte Linked Data Cloud\cite{lodcloud} verschiedenster Disziplinen (Geod{\"a}sie / Geisteswissenschaften / Naturwissenschaften), in der amtliche Geodaten als Bindeglied dienen, um neues Wissen f{\"u}r die Gesellschaft zu erzeugen\cite{florian_thiery_2019_2620929}.

\bibliographystyle{IEEEtran}
\bibliography{bib}

\end{document}